\documentclass[14pt,a4paper,report]{ncc}
\usepackage[a4paper, mag=1000, left=2.5cm, right=1cm, top=2cm, bottom=2cm, headsep=0.7cm, footskip=1cm]{geometry}
\usepackage[utf8]{inputenc}
\usepackage[english,russian]{babel}
\usepackage{indentfirst}
\usepackage[dvipsnames]{xcolor}
\usepackage[colorlinks]{hyperref}
\usepackage{listings} 
\usepackage{caption}
\usepackage{listings}
\usepackage{graphicx}


\DeclareCaptionFont{white}{\color{white}} 
\DeclareCaptionFormat{listing}{\colorbox{gray}{\parbox{\textwidth}{#1#2#3}}}
\captionsetup[lstlisting]{format=listing,labelfont=white,textfont=white}
\lstset{% Собственно настройки вида листинга
inputencoding=utf8, extendedchars=\true, keepspaces = true, % поддержка кириллицы и пробелов в комментариях
language=Java,            % выбор языка для подсветки (здесь это Pascal)
basicstyle=\small\sffamily, % размер и начертание шрифта для подсветки кода
numbers=left,               % где поставить нумерацию строк (слева\справа)
numberstyle=\tiny,          % размер шрифта для номеров строк
stepnumber=1,               % размер шага между двумя номерами строк
numbersep=5pt,              % как далеко отстоят номера строк от подсвечиваемого кода
backgroundcolor=\color{white}, % цвет фона подсветки - используем \usepackage{color}
showspaces=false,           % показывать или нет пробелы специальными отступами
showstringspaces=false,     % показывать или нет пробелы в строках
showtabs=false,             % показывать или нет табуляцию в строках
frame=single,               % рисовать рамку вокруг кода
tabsize=2,                  % размер табуляции по умолчанию равен 2 пробелам
captionpos=t,               % позиция заголовка вверху [t] или внизу [b] 
breaklines=true,            % автоматически переносить строки (да\нет)
breakatwhitespace=false,    % переносить строки только если есть пробел
escapeinside={\%*}{*)}      % если нужно добавить комментарии в коде
}
\begin{document}
% Переоформление некоторых стандартных названий
%\renewcommand{\chaptername}{Лабораторная работа}
\def\contentsname{Содержание}




% Оформление титульного листа
\begin{titlepage}
\begin{center}
\textsc{СПБ НИУ ИТМО \\[2mm]«Университет информационных технологий,\\ механики и оптики»\\[5mm]
Кафедра вычислительной техники\\[2mm]}


\vfill

\textbf{ОТЧЁТ ПО ЛАБОРАТОРНОЙ РАБОТЕ №3\\[3mm]
Вариант №505 \\[25mm]
}
\end{center}

\hfill
\begin{minipage}{.3\textwidth}
Выполнили:  \\[2mm] 
Ощепков А.А.\\
Буланцов А.М. \\
группа: P3202\\[5mm]

Преподаватель:\\[2mm] 
Николаев В.В.
\end{minipage}%
\vfill
\begin{center}
Санкт-Петербург \theyear\
\end{center}
\end{titlepage}

\newpage
\large \textbf{Задание} 
\small
\par

Разработать приложение на базе JavaServer Faces Framework, которое осуществляет проверку попадания точки в заданную область на координатной плоскости.

Приложение должно включать в себя 2 facelets-шаблона - стартовую страницу и основную страницу приложения, а также набор управляемых бинов (managed beans), реализующих логику на стороне сервера.

Стартовая страница должна содержать следующие элементы:

"Шапку", содержащую ФИО студента, номер группы и номер варианта.
Интерактивные часы, показывающие текущие дату и время, обновляющиеся раз в 10 секунд.
Ссылку, позволяющую перейти на основную страницу приложения.
Основная страница приложения должна содержать следующие элементы:

Набор компонентов для задания координат точки и радиуса области в соответствии с вариантом задания. Может потребоваться использование дополнительных библиотек компонентов - ICEfaces (префикс "ace") и PrimeFaces (префикс "p"). Если компонент допускает ввод заведомо некорректных данных (таких, например, как буквы в координатах точки или отрицательный радиус), то приложение должно осуществлять их валидацию.
Динамически обновляемую картинку, изображающую область на координатной плоскости в соответствии с номером варианта и точки, координаты которых были заданы пользователем. Клик по картинке должен инициировать сценарий, осуществляющий определение координат новой точки и отправку их на сервер для проверки её попадания в область. Цвет точек должен зависить от факта попадания / непопадания в область. Смена радиуса также должна инициировать перерисовку картинки.
Таблицу со списком результатов предыдущих проверок.
Ссылку, позволяющую вернуться на стартовую страницу.
Дополнительные требования к приложению:

Все результаты проверки должны сохраняться в базе данных под управлением СУБД Oracle.
Для доступа к БД необходимо использовать протокол JDBC без каких-либо дополнительных библиотек.
Для управления списком результатов должен использоваться Application-scoped Managed Bean.
Конфигурация управляемых бинов должна быть задана с помощью параметров в конфигурационном файле.
Правила навигации между страницами приложения должны быть заданы в отдельном конфигурационном файле.
\\[2mm]
\large \textbf{Текст программы}\\
\small

\begin{lstlisting}public class AreaInteractionController {

    private MainModel model;

    @ManagedProperty("point")
    private Point point;

    @PostConstruct
    void init(){
//        FacesContext fc = FacesContext.getCurrentInstance();
//        model = fc.getApplication().evaluateExpressionGet(fc,"#{model}",MainModel.class);
        point = new Point(0, 0, 0.1);
    }

    public AreaInteractionController(){

    }
    public Point getPoint() {
        return point;
    }

    public void setPoint(Point point) {
        this.point = point;
    }

    public void handle_sususu() throws Exception{
        model.addPoint(point);
    }

    public MainModel getModel() {
        return model;
    }

    public void setModel(MainModel model) {
        this.model = model;
    }
}
\end{lstlisting}

\begin{lstlisting}
<?xml version='1.0' encoding='UTF-8'?>
<faces-config version="2.2" xmlns="http://xmlns.jcp.org/xml/ns/javaee"
    xmlns:xsi="http://www.w3.org/2001/XMLSchema-instance"
    xsi:schemaLocation="http://xmlns.jcp.org/xml/ns/javaee 
    http://xmlns.jcp.org/xml/ns/javaee/web-facesconfig_2_2.xsd">

    <managed-bean>
        <managed-bean-name>clock</managed-bean-name>
        <managed-bean-class>smt.Beans.ClockController</managed-bean-class>
        <managed-bean-scope>session</managed-bean-scope>
    </managed-bean>

    <managed-bean>
        <managed-bean-name>model</managed-bean-name>
        <managed-bean-class>smt.Beans.MainModel</managed-bean-class>
        <managed-bean-scope>application</managed-bean-scope>
        <managed-property>
            <property-name>database</property-name>
            <value>#{dbImpl}</value>
        </managed-property>
    </managed-bean>
    ...
    
\end{lstlisting}

\large \textbf{Выводы}
\small
\par
В результате выполнения данной лабораторной работы, мы познакомились с технологией JSF, впервые на практике использовали DI, сделали ещё один шажок в болота энтерпрайза.\\

\end{document}
